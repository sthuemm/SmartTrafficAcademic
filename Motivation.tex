\section{Motivation}

Im heutigen Zeitalter lassen sich in allen Bereichen des Lebens situationsbezogene Daten gewinnen. Hinter dem Schlagwort Big Data verbirgt sich die Idee aus gesammelten Daten Schlüsse zu ziehen. Im Allgemeinen entsteht daraus ein Potential verschiedenste Situationen  besser einzuordnen. Es lassen sich komplexe Zusammenhänge analysieren und effiziente Herangehensweisen entwickeln.  Ansätze des Maschinellen Lernens, des Autonomen Fahrens oder auch Vorhersagen von Aktienkursen und des Wetters basieren auf Erkenntnissen historisch erhobener Datenmengen. Es dreht sich alles um die Frage, welche Zusammenhänge und Muster zu welchen Aktionen und Interaktionen führen.
Ein Paradebeispiel aus dem Alltag findet man im Straßenverkehr. Unter dem Begriff \emph{Smart Traffic} versteht man eine intelligente Vernetzung von Verkehrsteilnehmern und Verkehrskomponenten wie beispielsweise Ampelschaltungen an Kreuzungen. Mithilfe von Echtzeit Datenerhebungen zur Verkehrsdichte, Umweltfaktoren oder speziellen Ereignissen, wie einem Unfall, sollen Verkehrsflüsse gesteuert und umgeleitet werden. Der Dateninput entsteht durch Sensoren an Straßenrändern oder durch interagierende Systeme in den Autos, welche zum Beispiel die Information über die Position und das Ziel der Verkehrsteilnehmer kommunizieren. Die verschiedenen Informationen werden zentral an einer Stelle zusammengeführt und analysiert. Aus historisch gewonnen Erkenntnissen zum Verkehrsverhalten bei Staus, Umweltbelastungen oder Unfällen lassen sich nun effiziente Reaktionen auf solche Szenarien anstoßen.  Mit der Ereignisverarbeitung ist  es dann möglich Verkehrsströme umzuleiten, in dem Ampeln verkehrsgerecht umgeschaltet werden und die Zielführung der Teilnehmer angepasst wird. Ziel ist ein intelligentes Netz zur optimalen Steuerung des Verkehrsaufkommens in bestimmten Zonen. Eine Echtzeit-Ereignisverarbeitung für einen flüssigen Verkehrsstrom bietet dabei nicht nur den Verkehrsteilnehmern einen großen Vorteil. Im Falle eines Unfalls kann die Umleitung des Verkehrs eine verbesserte Versorgung durch die Einsatzkräfte sichergestellt werden. Ein aktuelles Thema in der Politik Deutschlands ist die Umweltverschmutzung durch ein zu hohes Verkehrsaufkommen. Besonders in Stuttgart kommt es regelmäßig zu sogenannten Feinstaubalarmen. Mit Sensoren am Fahrbahnrand in den beeinträchtigten Regionen lässt sich die Umweltbelastung detektieren. Bei einem erhöhten Schadstoffgehalt in der Luft ist eine Umleitung des Verkehrs ebenfalls eine geeignete Maßnahme. Von einer solchen Steuerung profitiert die gesamte (einheimische) Gesellschaft. 
Smart Traffic umreißt also einen spannenden Ansatz mit vielen Facetten zur Verbesserung der Gesamtsituation im Straßenverkehr und der Umwelt. Die Idee beruht im Grunde darauf mit Hilfe einer Mustererkennung verschiedene Szenarien im Straßenverkehr zu identifizieren und eine automatisierte, sowie optimale Reaktion anzustoßen. In dieser Arbeit wird die Umsetzung eines fiktiven, aber realitätsnahen, Fallbeispiels für eine ereignisbasierte Verkehrssteuerung beschrieben. 


\clearpage
