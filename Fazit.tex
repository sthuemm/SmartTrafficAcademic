\section{Fazit}

Diese Arbeit entstand im Kontext des Studienfachs Data Analytics. Im Fokus steht die Technologie des Complex Event Processing (CEP). In allen Lebensbereichen in denen bestimmte Muster erkannt und entsprechende Aktionen an getriggert werden müssen, sind CEP Ansätze eine sehr gute Lösungsmöglichkeit. Wie vielseitig die Möglichkeiten sind, wurde im Laufe der Veranstaltung durch die Vorstellung verschiedenster CEP Einsatzbereiche deutlich. Kenntnisse über die Einsatzgebiete, Umsetzungen und Alternativen können im Berufsleben eines Informatikers eine wertvolle Kompetenz sein. In diesem Zusammenhang wurde in dieser Arbeit der Einsatz von CEP Engins im Kontext von Smart Traffic bearbeitet.
Im Kapitel… wurde der Leser an Probleme unserer Zeit herangeführt, welche durch das kontinuierlich wachsende Verkehrsaufkommen der letzten Jahre bedingt sind. Neben der Umweltbelastung durch die Schadstoffemissionen, wurde der negative Wirtschaftsfaktor beschrieben und ansatzweise auch die Auswirkungen von Stauwartezeiten auf das menschliche Stressniveau. Eine Smart Traffic Lösung ist ganz sicher ein innovativer Schritt um eine allgemeine Verbesserung der Gesellschaftlichen Lebensqualität zu erreichen. Vorhandene technologische Entwicklungen des Computerzeitalters werden gewinnbringend eingesetzt. Auf den ersten Blick haben Smart Traffic Systeme ausschließlich positive Effekte auf Umwelt, Gesellschaft und Wirtschaft. In Kapitel… wurden aber auch berechtigte Kritikpunkte aufgezeigt. Man muss sich den Herausforderungen zum digitalen Sicherheitsniveau, sowie der Frage nach rechtlichen Verbindlichkeiten stellen. Zudem ist eine Smart Traffic Umsetzung beziehungsweise der benötigte Infrastrukturumfang mit hohen Kosten verbunden. Straßenverkehrsszenarien sind sehr komplex, da es überabzählbar  viele Ereignisse und Ereigniskombinationen gibt. Im Fallbeispiel dieser Arbeit wurde ein fiktiver Straßenausschnitt (siehe Abbildung…) betrachtet, auf dem die vier Ereignisse Unfall, Stickstoffsensoralarm, Bahnschranke und Verkehrsdichte auftreten können. Wenn man nun davon ausgeht, dass jedes Ereignis (englisch Event) entweder eintreten oder nicht eintreten kann, ergibt das mathematisch 16 (2 hoch 4) Ereigniskombinationen, für welche man eine Reaktion oder Gegenmaßnahme definieren muss. Das Szenario ist vielleicht noch überschaubar, doch es wird auch nur ein vergleichsweise kleiner Straßenausschnitt betrachtet. Überträgt man die Logik der Ereigniskombinationen in die reale Welt, so wird schnell klar, wie unüberschaubar komplex die Strukturen und Abhängigkeiten werden können. Bei realen Projekten bricht man die Idee einer Smart Traffic Lösung auf kleinere Einsatzgebiete oder Bereiche herunter. Wie in den Beispielen im Kapitel… aufgezeigt, werden mit Sensorsystemen Hauptverkehrsstraßen, Knotenpunkte oder Parkplätze detektiert. Nach und nach werden in den Städten weitere Komponenten in die Automatisierung der Verkehrsführung integriert.
Insgesamt bietet der Straßenverkehr zwar ein unheimlich komplexes, aber dennoch wohl-definiertes und geordnetes System.  Es lassen sich einzelne Komponenten der Verkehrsführung oder Straßenausschnitte  separat betrachten und zeitlich unabhängig zusammensetzen. Daraus lässt sich ein großes Potential für die Automatisierung im Sinne von Smart Traffic ableiten. Smart Traffic nutzt dabei alle Bereiche, die man zum Thema BigData gruppiert. Von der empirischen Datenerhebung, über die Musterentwicklung, wird schließlich die Mustererkennung im realen Betrieb eingesetzt und Präventive, sowie Reaktive Maßnahmen auf Verkehrsszenarien angestoßen.
In dieser Arbeit wurde ein realitätsnahes Fallbeispiel nach der Idee von Smart Traffic erarbeitet und umgesetzt. Als CEP Engine wurde das flexibel, konfigurierbare ESPER Framework verwendet. Dieses wurde in eine MVC  Softwarearchitektur, bestehend aus einem SpringBoot Backend und einer JSP  Frontendsteuerung, implementiert. Resümiert betrachtet lässt sich zweifelsohne feststellen, dass sich das Fallbeispiel hervorragend für den Einsatz einer CEP Engine eignet und ein wertvoller Lernerfolg erzielt wurde.



