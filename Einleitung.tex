\section{Einleitung}

Diese Ausarbeitung befasst sich mit dem Thema Smart Traffic. Im Modul Data Analytics des Masterstudiengangs Informatik an der HTWG sollen Möglichkeiten und Einsatzgebiete für das Complex Event Processing (CEP)  erarbeitet werden. Der Begriff \emph{Smart Traffic} bezeichnet dabei eine ereignisbasierte Mustererkennung im Straßenverkehr. Anhand einer Fallstudie werden in dieser Arbeit Szenarien für eine intelligente Steuerung des Verkehrs aufgezeigt. Abb. \ref{fig1} zeigt die Visualisierung  eines Straßenausschnitts, welchen wir für die Veranschaulichung der Smart Traffic Fallstudie verwenden. In diesem Verkehrskontext können nun fiktive Datenströme erzeugt werden, um bestimmte Verkehrssituationen zu simulieren. Eine Implementierung der CEP Engine ESPER erlaubt uns die Erkennung und Verarbeitung von Verkehrsszenarien. Wir generieren Datenströme, welche beispielsweise einen lokalen Unfall  repräsentieren und benutzen die Mustererkennung in ESPER, um eine kluge (englisch smart) Umleitung des Verkehrs anzustoßen. Mit Hilfe einer WebUI können verschiedene Kombinationen von Verkehrsereignissen in unsere SMART Traffic Applikation kommuniziert werden.

\begin{figure}[ht]
	\includegraphics[width=\textwidth]{images/1_InitialStreetMap_Final.png}
	\caption{Straßenausschnitt der Fallstudie}
	\label{fig1}
\end{figure}




